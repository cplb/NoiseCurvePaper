\section{Astrophysical sources}\label{sec:sources}

All the sources described here are represented by shaded boxes in \ref{app:a}. The boxes are drawn in such a way that there is a reasonable event rate for sources at a detectable SNR. Descriptions of each source are given in the following sections. Sources with short durations (i.e., burst sources) and sources that evolve in time over much longer timescales than our observations are drawn with flat-topped boxes for $h_\mathrm{c}(f)$. Inspiralling binaries, or stochastic backgrounds of binaries, are drawn with a sloping top proportional to $f^{-2/3}$ for $h_\mathrm{c}(f)$, which is result derived in section \ref{sec:stoch}. The width of the box gives the range of frequencies sources of a given type can have while remaining at a detectable amplitude.

\subsection{Sources for ground-based detectors}

\subsubsection{Neutron star binaries}

The inspiral and merger of a pair of neutron stars is the primary target for ground-based detectors. The expected event rate for this type of source is uncertain, but estimates centre around $\gamma_{\mathrm{NS-NS}}=1.3\times 10^{-4}~\mathrm{Mpc^{-3}\,yr^{-1}}$ \citep{CBC}. Plotted in \ref{app:a} are boxes labelled ``\emph{compact binary inspirals}'' with amplitudes such that a ratio $h_\mathrm{c}/h_{n}=16$ is produced for Advanced LIGO at peak sensitivity and a width between $(3$--$300)~\mathrm{Hz}$, corresponding to expected observable frequencies.

\subsubsection{Supernovae}

Simulations of core-collapse supernovae show that GWs between $(10^{2}$--$10^{3})~\mathrm{Hz}$ can be produced \citep{Kotake2006}. The GW signal undergoes ${\mathcal{O}}(1)$ oscillation and is hence burst-like. \citet{2002A&A...393..523D} calculate the average maximum amplitude of GWs for a supernova at distance $r$ as
\begin{equation}
h_\mathrm{max} = 8.9\times 10^{-21}\left( \frac{10~\mathrm{kpc}}{r} \right) \; .
\end{equation}
The event rate for supernovae is approximately $\gamma_{\mathrm{SN}} = 5\times10^{-4}~\mathrm{Mpc^{-3}\,yr^{-1}}$. The boxes labelled ``\emph{supernova}'' plotted in \ref{app:a} correspond to a distance $r=300~\mathrm{kpc}$ with the frequency range quoted above. The LIGO and Virgo detectors have already placed bounds on the event rate for these sources \citep{Bursts}.

\subsubsection{Continuous waves from rotating neutron stars}

Rotating neutron stars are a source of continuous GWs if they possess some degree of axial asymmetry \citep{Abbott2007, Prix2009, Einstein@Home}. The signals are near monochromatic with a frequency twice the rotation frequency of the neutron star, and are a potential source for ground-based detectors. The amplitude of the GWs depends upon the deformation of the neutron star, which itself depends upon the neutron star equation of state. This is currently uncertain \citep{Lattimer2012}. Several known pulsars could be sources for the advanced detectors and upper limits from the initial detectors help to constrain the deformations. The boxes labelled ``\emph{pulsars}'' plotted in \ref{app:a} correspond to the upper limits placed on a GW signal from the Crab pulsar \citep{Aasi2014a}, extrapolated across a frequency range between $(20$--$10^{3})~\mathrm{Hz}$.

\subsection{Sources for space-based detectors}

For a review of the GW sources for space-based missions see, for example, \citet{Amaro-Seoane-et-al}, \citet{Gairetal} or \citet{eLISAyellowbook}.


\subsubsection{Massive black hole binaries}

Space-based detectors shall be sensitive to equal-mass mergers in the range $(10^{4}$--$10^{7})\,\Msun$. Predictions of the event rate for these mergers range from ${\mathcal{O}}(10$--$100)~\mathrm{yr}^{-1}$ for eLISA with SNRs of up to $10^3$ \citep{TheGravitationalUniverse}. The large range in the rate reflects our uncertainty in the growth mechanisms of the supermassive black hole population \citep{Volonteri2010}. Plotted in \ref{app:a} are boxes labelled ``\emph{$\approx 10^{6}$ solar mass binaries}'' with a ratio $h_\mathrm{c}/h_{n}=100$ for eLISA at its peak sensitivity. The range of frequencies plotted is $(3\times 10^{-4}$--$3\times 10^{-1})~\mathrm{Hz}$; this corresponds to circular binaries in the mass range quoted above.

\subsubsection{Galactic white dwarf binaries} \label{sec:GB}

For space-based detectors, these are the most numerous GW sources; they are also the only guaranteed source since several detectable systems (known as verification binaries) have already been identified by electromagnetic observations \citep{2006CQGra..23S.809S}.

Galactic binaries divide into two classes: the unresolvable and the resolvable galactic binaries. The unresolvable binaries overlap to form a stochastic background as discussed in section \ref{sec:stoch}. The distinction between resolvable and unresolvable is detector specific; here we choose LISA. This boundary will not be too different for eLISA but would move substantially for either of the decihertz detectors. Plotted in \ref{app:a} with the label ``\emph{unresolvable galactic binaries}'' is the estimate of this background due to \citet{Nelemans} where an observation time of one year has been assumed,
\begin{equation}
h_\mathrm{c}(f)= 5\times 10^{-21} \left(\frac{f}{10^{-3}~\mathrm{Hz}}\right)^{-2/3} \; .
 \end{equation}
Estimates for the event rate of resolvable binaries centre around ${\mathcal{O}}(10^{3})$ events for eLISA. The boxes plotted in \ref{app:a} with the label ``\emph{resolvable galactic binaries}'' have a ratio $h_\mathrm{c}/h_{n}=50$ for eLISA at its peak sensitivity. The frequency range of the box is $\left(3\times10^{-4}\right.$--$\left.10^{-2}\right)~\mathrm{Hz}$, estimated from Monte Carlo population simulation results presented in \citet{Amaro-Seoane-et-al}.

\subsubsection{Extreme mass-ratio inspirals}

EMRIs occur when a compact stellar mass object inspirals into a supermassive black hole. There is extreme uncertainty in the event rate for EMRIs due to the poorly constrained astrophysics in galactic centres \citep[e.g.,][]{Merritt2011}; the best guess estimate is around $25$ events per year with eLISA with SNR $\ge 20$ \citep{TheGravitationalUniverse}. The boxes labelled ``\emph{extreme mass ratio inspirals}'' plotted in \ref{app:a} have a characteristic strain of $h_\mathrm{c}=3\times 10^{-20}$ at $10^{-2}~\mathrm{Hz}$, which corresponds to a $10\Msun$ black hole inspiralling into a $10^{6}\Msun$ black hole at a luminosity distance of $1~\mathrm{Gpc}$. The frequency width of the box is somewhat unknown; EMRI events can occur into a black hole of any mass, and hence EMRIs can, in principal, occur at any frequency, so the boxes in \ref{app:a} are drawn with a width comparable to that of the LISA sensitivity curve.




\subsection{Sources for PTAs}

\subsubsection{Supermassive black hole binaries}

The main target for PTAs is a stochastic background of GWs produced by a population of supermassive black hole binaries at cosmological distances \citep{SesanaVecchioColancino}. Supermassive black holes are known to lie at the centres of most galaxies and the black hole mergers are associated with the mergers of the host galaxies \citep{Volonteri2003,Ferrarese2005}. The current best published limit for the amplitude of the stochastic background is $h_\mathrm{c} = 6\times 10^{-15}$ at a frequency of $f_{0}=1~\mathrm{yr}^{-1}$ \citep{Haasteren}. There is strong theoretical evidence that the actual background lies close to the current limit \citep{Sesana-2012}. 

Supermassive black hole binaries at higher frequencies are inspiralling faster and hence there are fewer of them per frequency bin. At a certain frequency, these sources will cease to be a background and become individually resolvable. It is currently unclear whether PTAs will detect an individual binary or a stochastic background first. Plotted in \ref{app:a} with the label ``\emph{stochastic background}'' is a third of the current limit with a cut off frequency of $f=1~\mathrm{yr}^{-1}$ which is suggested by Monte Carlo population studies \citep{SesanaVecchioColancino}. For the resolvable sources, labelled ``\emph{$\approx 10^{9}$ solar mass binaries}'', the amplitude of the current limit is plotted between $\left(3\times 10^{-9}\right.$--$\left.3\times 10^{-7}\right)~\mathrm{Hz}$.




\subsection{Cosmological sources}

In addition to the sources above, early Universe processes, such as inflation \citep{Grishchuk2005} or a first-order phase transition \citep{Binetruy2012}, could have created GWs. More speculatively, it has been hypothesised that cosmic strings could also be a potential source \citep{Damour2005,Binetruy2012,Aasi2014}. These relic GWs allow us to explore energy scales far beyond those accessible by other means, providing insight into new and exotic physics. The excitement surrounding the tentative discovery by BICEP2 of the imprint of primordial GWs (generated during inflation) in the cosmic microwave background \citep{Ade2014}, and the subsequent flurry of activity, has shown the scientific potential of such cosmological GWs. These GW signals are so alluring because they probe unknown physics; this also makes them difficult to predict. Cosmological stochastic backgrounds have been predicted across a range of frequencies with considerable variation in amplitude. As a consequence of this uncertainty, although we could learn much from measuring these signals, we have not included them amongst the sources in \ref{app:a}.


