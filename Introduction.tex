\section{Introduction}

%CPLB: alternate openning I put together a while ago.
%Even the best of us can be frustrated by the use of different conventions or units when describing a single quantity. It is a common occurrence in physics that  there is a particular parameterization that is perfect for one experiment or calculation, but this is not of universal applicability. When comparing results across studies it is then necessary to trawl through the literature to check that $x$ is really the same parameter. Astronomy and astrophysics are rife with specialized units, from the jansky (a spectral flux density of $10^{-26}$ watts per square metre per hertz) to the solar neutrino unit (a neutrino flux that produces an interaction rate of $10^{-36}$ per second per target atom). Inevitably, things get lost in translation. The mismatch of conventions can be confusing to those new to the field, especially students.

The next few years  promise to deliver the first direct detection of gravitational waves (GWs). This will most likely be achieved by the advanced Laser Interferometer Gravitational-wave Observatory \citep[LIGO;][]{2010CQGra..27h4006H} and advanced Virgo \citep{Acernese2009} detectors operating in the frequency range $(10$--$10^{3})~\mathrm{Hz}$. By the end of the decade, it is expected that pulsar timing arrays \citep[PTAs;][]{1990ApJ361300F} will also detect very low frequency GWs around $10^{-8}~\mathrm{Hz}$. Further into the future, space-based detectors, such as the evolved Laser Interferometer Space Antenna \citep[eLISA;][]{TheGravitationalUniverse}, will probe GWs in the millihertz regime. These advances shall herald the beginning of multi-wavelength GW astronomy as a means of observing the Universe.

There already exists an extensive literature assessing the potential of all of these detectors to probe the astrophysics of various sources. There are several different methods commonly used to describe the sensitivity of a GW detector and the strength of a GW source. It is common practice to summarise this information on a sensitivity-curve plot. When producing these plots, it is desirable to have a consistent convention between detectors and sources that allows information about both to be plotted on the same graph. Ideally, the detectors and sources are represented in such a way that the relative detectability of the signals is immediately apparent from the plot.

In this work, we discuss the differing conventions commonly used in GW astronomy. The amplitude of a GW is a strain, a dimensionless quantity $h$. This gives a fractional change in length, or equivalently light travel time, across a detector. The strain is small, making it a challenge to measure: we are yet to obtain a direct detection of a GW. To calibrate our expectations for future detections it is necessary to quantify the sensitivity of our instruments and the strength of their target signals. When discussing the loudness of sources and the sensitivity of detectors there are three commonly used parametrizations based upon the strain: the characteristic strain, the power spectral density (PSD) and the spectral energy density. We aim to disambiguate these three and give a concrete comparison of different detectors. It is hoped that this will provide a useful reference for new and experienced researchers in this field alike. 

We begin by expounding the various conventions and the relationships between the conventions in sections \ref{sec:conventions} and \ref{sec:voc}. A review of GW detectors (both current and proposed) is given in section \ref{sec:detectors} and a review of GW sources is given in section \ref{sec:sources}. In \ref{app:a} several example sensitivity curves are presented. A website where similar figures can be generated is available at \url{www.ast.cam.ac.uk/~rhc26/sources/}. Here the user may select which sources and detectors to include to tailor the figure to their specific requirements.


\section{Signal parametrization}\label{sec:conventions}

\subsection{Signal analysis preliminaries}

Gravitational radiation has two independent polarization states denoted $+$ and $\times$; a general signal can be described as a linear combination of the two polarization states, $h \nobreak=\nobreak A_{+} h_{+}\nobreak +\nobreak A_{\times} h_{\times}$. The sensitivity of a detector to these depends upon the relative orientations of the source and detector. The output of a gravitational wave detector $s(t)$ contains a superposition of noise $n(t)$ and (possibly) a signal $h(t)$,
\begin{equation}
s(t) = n(t)+h(t) \; .
\end{equation}
We shall have recourse to work with the Fourier transform of the signal, using the conventions that
\begin{eqnarray} \label{eq:defFourier}
\tilde{x}(f) &= {\mathcal{F}}\left\{x(t)\right\}(f) &= \int_{-\infty}^{\infty}\mathrm{d}t\;x(t)\exp (-2\pi \rmi f t ) \; , \\
x(t) &= {\mathcal{F}}^{-1}\left\{\tilde{x}(f)\right\}(t) &= \int_{-\infty}^{\infty}\mathrm{d}f\;\tilde{x}(f)\exp (2\pi \rmi f t ) \; .
\end{eqnarray}

For simplicity, it is assumed that the noise in the GW detector is stationary and Gaussian (with zero mean); under these assumptions the noise is fully characterised via the one-sided noise PSD $S_{n}(f)$, 
\begin{equation}\label{eq:psd}
\left<\tilde{n}(f)\tilde{n}^{*}(f')\right>=\frac{1}{2}\delta (f-f')S_{n}(f) \; ,
\end{equation}
where angle brackets $\left<\ldots\right>$ denote an ensemble average over many noise realisations \citep{Cutler1994}. In reality, we have only a single realisation to work with, however the ensemble average can be replaced by a time average for stationary stochastic noise. The procedure is to measure the noise over a sufficiently long duration $T$ and then compute the Fourier transform $\tilde{n}(f)$ with a frequency resolution $\Delta f = T^{-1}$; this is repeated many times to give an average. The noise PSD $S_{n}(f)$ has units of inverse frequency.

Since the GW signal and detector output are both real it follows that $\tilde{h}(-f)=\tilde{h}^{*}(f)$ and $\tilde{n}(-f)=\tilde{n}^{*}(f)$; therefore, $S_{n}(f)=S_{n}(-f)$. The fact that $S_{n}(f)$ is an even function means that Fourier integrals over all frequencies can instead be written as integrals over positive frequencies only, e.g., (\ref{eq:meansquare1}) and (\ref{eq:snrinnerprod}); it is for this reason that $S_{n}(f)$ is called the \emph{one-sided} PSD.\footnote{An alternative convention is to use the two-sided PSD $S^{(2)}_{n}(f) = S_{n}(f)/2$.}

When integrated over all positive frequencies, the PSD gives the mean square noise amplitude. Starting by taking the time average of the square of the detector noise:
\begin{eqnarray}
\overline{\left|n(t)\right|^{2}} &= \lim_{T\rightarrow\infty}\frac{1}{2T} & \int_{-T}^{T}\mathrm{d}t\; n(t)n^{*}(t) \\
 &= \lim_{T\rightarrow\infty}\frac{1}{2T} & \int_{-T}^{T}\mathrm{d}t\; \int_{-\infty}^{\infty}\mathrm{d}f\;\int_{-\infty}^{\infty}\mathrm{d}f'\;\tilde{n}(f)\tilde{n}^{*}(f')\exp\left(2\pi \rmi ft\right)\exp\left(-2\pi \rmi f't\right) \nonumber\\
 &= \lim_{T\rightarrow\infty}\frac{1}{2T} & \int_{-T}^{T}\mathrm{d}t\; \int_{-\infty}^{\infty}\mathrm{d}f\;\int_{-\infty}^{\infty}\mathrm{d}f'\;{\mathcal{F}}\left\{ n(\tau) \right\}(f)\left[{\mathcal{F}}\left\{ n(\tau) \right\}(f')\right]^{*} \nonumber \\*
\label{eq:meansquare} & & \times \exp\left(2\pi \rmi ft\right)\exp\left(-2\pi \rmi f't\right)\;,
\end{eqnarray}
where we have substituted in using the definitions of the Fourier transform and its inverse. A property of Fourier transforms is that a time-domain translation by amount $t$ is equivalent to a frequency-domain phase change $2\pi ft$; if ${\mathcal{F}}\left\{ n(\tau) \right\}(f) = \tilde{n}(f)$, then ${\mathcal{F}}\left\{ n(\tau-t) \right\}(f) = \tilde{n}(f)\exp (2\pi \rmi ft )$. Therefore, the exponential factors in (\ref{eq:meansquare}) may be absorbed as
\begin{eqnarray}
\overline{\left|n(t)\right|^{2}} &=& \lim_{T\rightarrow\infty}\frac{1}{2T}\int_{-T}^{T} \mathrm{d}t\:\int_{-\infty}^{\infty}\mathrm{d}f\;\int_{-\infty}^{\infty}\mathrm{d}f'\;{\mathcal{F}}\left\{ n(\tau-t) \right\}(f)\left[{\mathcal{F}}\left\{ n(\tau-t) \right\}(f')\right]^{*} \, .
\end{eqnarray}
Since the noise is a randomly varying signal, we can use the ergodic principle to equate a time average, denoted by $\overline{\left(\ldots\right)}$, with an ensemble average, denoted by $\left<\ldots\right>$. The noise is stationary, consequently, its expectation value is unchanged by the time-translation performed above. Therefore, using (\ref{eq:psd}), the mean square noise amplitude is given by
\begin{eqnarray}\label{eq:meansquare1}
\overline{\left|n(t)\right|^{2}} &=& \int_{-\infty}^{\infty}\mathrm{d}f\;\int_{-\infty}^{\infty}\mathrm{d}f'\;\left<\tilde{n}(f)\tilde{n}^{*}(f')\right> \\
&=& \int_{-\infty}^{\infty}\mathrm{d}f\;\int_{-\infty}^{\infty}\mathrm{d}f'\;\frac{1}{2}S_{n}(f)\delta(f-f') \nonumber \\
&=& \int_{0}^{\infty}\mathrm{d}f\; S_{n}(f).
\end{eqnarray}

Given a detector output, the challenge is to extract the signal. There is a well known solution to this problem that involves constructing a Wiener optimal filter \citep{Wiener49}. Let $K(t)$ be a real filter function with Fourier transform $\tilde{K}(f)$. Convolving this with the detector output gives a contribution from the signal and a contribution from the noise,
\begin{equation}\label{eq:conv}
\left(s*K\right)(\tau) = \int_{-\infty}^{\infty}\mathrm{d}t\;\left[h(t)+n(t)\right]K(t-\tau) \approx \mathcal{S} + \mathcal{N} \; .
\end{equation}
The signal contribution $\mathcal{S}$ is defined as the expectation of the convolution in (\ref{eq:conv}) when a signal is present, maximised by varying the offset to achieve the best overlap with the data. Since the expectation of pure noise is zero it, follows that
\begin{equation}
\mathcal{S} = \int_{-\infty}^{\infty}\mathrm{d}t\;h(t)K(t)=\int_{-\infty}^{\infty}\mathrm{d}t\;h(t)K^{*}(t)=\int_{-\infty}^{\infty}\mathrm{d}f\; \tilde{h}(f)\tilde{K}^{*}(f) \; .
\end{equation}
The squared contribution from noise $\mathcal{N}^{2}$ is defined as the mean square of the convolution in (\ref{eq:conv}) when no signal is present,
\begin{eqnarray} 
\mathcal{N}^{2} &= \int_{-\infty}^{\infty}\mathrm{d}t\;\int_{-\infty}^{\infty}\mathrm{d}t'\;K(t)K(t')\left<n(t)n(t')\right> \nonumber \\
 &= \int_{-\infty}^{\infty}\mathrm{d}t\;\int_{-\infty}^{\infty}\mathrm{d}t'\;K(t)K^{*}(t')\int_{-\infty}^{\infty}\mathrm{d}f\;\int_{-\infty}^{\infty}\mathrm{d}f'\;\left<\tilde{n}(f)\tilde{n}^{*}(f')\right>\exp\left[2\pi\rmi(ft-f't')\right]\nonumber \\
%&=\int_{-\infty}^{\infty}\mathrm{d}f\;\int_{-\infty}^{\infty}\mathrm{d}t\;\int_{-\infty}^{\infty}\mathrm{d}t'\;\frac{1}{2}S_{n}(f)K(t)K^{*}(t')\exp\left(2\pi\rmi f (t-t')\right)\nonumber \\
 &= \int_{-\infty}^{\infty}\mathrm{d}f \; \frac{1}{2}S_{n}(f)\tilde{K}(f)\tilde{K}^{*}(f) \; ,
 \end{eqnarray}
using the definition of $S_{n}(f)$ from (\ref{eq:psd}). Hence the signal-to-noise ratio (SNR) $\varrho$ is given by
\begin{equation}\label{eq:SNRinnerprod} 
\varrho^{2} = \frac{\mathcal{S}^{2}}{\mathcal{N}^{2}}= \frac{\innerprod{\frac{1}{2}S_{n}(f)\tilde{K}(f)}{\tilde{h}(f)}^{2}}{\innerprod{\frac{1}{2}S_{n}(f)\tilde{K}(f)}{\frac{1}{2}S_{n}(f)\tilde{K}(f)}},
\end{equation}
where we have introduced the inner product between signal $\tilde{A}$ and $\tilde{B}$ as \citep{Finn1992}
\begin{equation}\label{eq:snrinnerprod} \innerprod{\tilde{A}(f)}{\tilde{B}(f)} = 4\Re\left\{\int_{0}^{\infty}\mathrm{d}f\;\frac{\tilde{A}^{*}(f)\tilde{B}(f)}{S_{n}(f)}\right\} \; .\end{equation}
The optimum filter is that function $\tilde{K}(f)$ which maximises the SNR in (\ref{eq:SNRinnerprod}). From the Cauchy--Schwarz inequality, it follows that the optimum filter is
\begin{equation}
\tilde{K}(f)=\frac{\tilde{h}(f)}{S_{n}(f)} \; .
\end{equation}
This is the Wiener filter, which may be multiplied by an arbitrary constant since this does not change the SNR. Using this form for $\tilde{K}(f)$, the squared SNR is
\begin{equation}
\varrho^2 = \int_0^\infty\mathrm{d}f \frac{4| \tilde{h}(f)|^{2}}{S_n(f)} = \innerprod{\tilde{h}(f)}{\tilde{h}(f)}.
\label{eq:traditionalSNR} 
\end{equation}
In order to construct the Wiener filter, it is necessary to know \emph{a priori} the form of the signal, $\tilde{h}(f)$, for this reason the Wiener filter is sometimes called the \emph{matched} filter.

Whilst the magnitude of the Fourier transform of the signal $|\tilde{h}(f)|$ provides a simple quantification of the GW amplitude as a function of frequency, it has one main deficiency. For an inspiralling source the instantaneous amplitude can be orders of magnitude below the noise level in a detector; however, as the signal continues over many orbits, the SNR can be integrated up to a detectable level. It is useful to have a quantification of the GW amplitude that accounts for this effect; three such methods are described in the following subsections.

\subsection{Characteristic strain}\label{sec:character-strain}

The characteristic strain $h_\mathrm{c}$ is designed to include the effect of integrating an inspiralling signal. Its counterpart for describing noise is the noise amplitude $h_n$. These are defined as
\begin{eqnarray}\label{eq:strain-hc} 
h_\mathrm{c}(f)^{2} &= 4f^{2}\left| \tilde{h}(f) \right|^{2} \; ,\\
h_{n}(f)^{2} &= fS_{n}(f),
\label{eq:strain-hn}
\end{eqnarray}
such that the SNR in (\ref{eq:traditionalSNR}) may be written
\begin{equation}\label{eq:hc} 
\varrho^{2} = \int_{0}^{\infty} \mathrm{d}\left(\log f\right)\; \left[\frac{h_\mathrm{c}(f)}{h_{n}(f)}\right]^{2} \;.
\end{equation}
The strain amplitudes $h_\mathrm{c}(f)$ and $h_{n}(f)$ are dimensionless. Using this convention, when plotting on a log--log scale, the area between the source and detector curves is related to the SNR via (\ref{eq:hc}). This convention allows the reader to integrate by eye to assess the detectability of a given source (see figure \ref{fig:hc}).

An additional advantage of this convention is that the values on the strain axis for the detector curve $h_n(f)$ have a simple physical interpretation: they correspond to the root-mean-square noise in a bandwidth $f$. One downside to plotting characteristic strain is that the values on the strain axis $h_\mathrm{c}(f)$ do not directly relate to the amplitude of the waves from the source. Another disadvantage is that the characteristic strain for a monochromatic source is infinite.

\subsection{Power spectral density }\label{sec:psd}

A second commonly used quantity for sensitivity curves is the square root of the PSD (see figure \ref{fig:S}). When discussing a detector this is
\begin{equation}\label{eq:temp1}
\sqrt{S_{n}(f)} = h_{n}(f)f^{-1/2} \; ,
\end{equation}
where we have used (\ref{eq:strain-hn}); by analogy, we can define an equivalent for source amplitudes
\begin{equation}
\sqrt{S_{h}(f)} = h_\mathrm{c}(f)f^{-1/2} = 2 f^{1/2} \left| \tilde{h}(f) \right| \; ,
\label{eq:ShforSources}
\end{equation}
where we have used (\ref{eq:strain-hc}). Both $\sqrt{S_{n}(f)}$ and $\sqrt{S_{h}(f)}$ have units of $\mathrm{Hz^{-1/2}}$. The root PSD is the most frequently plotted quantity in the literature.

The PSD has the nice property, demonstrated in (\ref{eq:meansquare1}), that integrated over all positive frequencies it gives the mean square amplitude of the signal in the detector. However, in one important regard it is less appealing than characteristic strain: the height of the source above the detector curve is no longer directly related to the SNR.


\subsection{Energy density}\label{sec:energy-density}

A third way of describing the amplitude of a GW is through the energy carried by the waves. This has the advantage of having a clear physical significance. The energy density is most commonly used in sensitivity curves showing stochastic backgrounds of GWs (see section \ref{sec:stoch}).

The energy in GWs is described by the Isaacson stress--energy tensor \citep[section 35.15]{MTW}
\begin{equation}
T_{\mu\nu}=\frac{c^{4}}{32\pi G}\left<\partial_{\mu}\bar{h}_{\alpha\beta}\partial_{\nu}\bar{h}^{\alpha\beta}\right> \;,
\end{equation}
where the angle brackets denote averaging over several wavelengths or periods and $\bar{h}_{\alpha\beta}$ is the transverse-traceless metric perturbation. The energy density $\rho c^{2}$ is given by the $T_{00}$ component of this tensor. Consequently \citep[cf.][]{Berry2013},
\begin{eqnarray}
\label{eq:specNRGdensity}
\rho c^{2} &=& \frac{c^{2}}{16\pi G}\int_{-\infty}^{\infty}\mathrm{d}f\;\left(2\pi f\right)^{2}\tilde{h}(f)\tilde{h}^{*}(f) \\*
 &=& \int_{0}^{\infty}\mathrm{d}f\;\frac{\pi c^{2}}{4G}f^{2}S_{h}(f)\; ,
\end{eqnarray} 
where the definition (\ref{eq:ShforSources}) have been used. Equation (\ref{eq:specNRGdensity}) gives the definition of the spectral energy density, the energy per unit volume of space, per unit frequency \citep{HellingsDowns}
\begin{equation}\label{eq:spectralenergydensity}
S_{\mathrm{E}}(f)=\frac{\pi c^{2}}{4G} f^{2}S_{h}(f) \; ;
\end{equation}
a corresponding expression for the noise can be formulated by replacing $S_h(f)$ with $S_{n}(f)$.

Cosmological studies often work in terms of the dimensionless quantity $\Omega_{\mathrm{GW}}$, the energy density per logarithmic frequency interval normalised to the critical density of the universe, $\rho_{\mathrm{c}}$:
\begin{equation}
\label{eq:omega}
\Omega_\mathrm{GW}(f) = \frac{fS_{\mathrm{E}}(f)}{\rho_{\mathrm{c}}c^{2}} \; , \quad \textrm{where}\quad \rho_{\mathrm{c}}=\frac{3H_{0}^{2}}{8\pi G} \;.
\end{equation}
where $H_{0}$ is the Hubble constant which is commonly parametrized as
\begin{equation}
H_0 =h_{100}\times 100~\mathrm{km\,s^{-1}\,Mpc^{-1}}.
\end{equation}
The reduced Hubble parameter $h_{100}$ has nothing to do with strain! The most common quantity related to energy density to be plotted on sensitivity curves is $\Omega_{\mathrm{GW}}h_{100}^{2}$ (figure \ref{fig:omega}) as this removes sensitivity to the historically very uncertain measured value of the Hubble constant.

This quantity has one aesthetic advantage over the others: it automatically accounts for the fact that there is less energy in low frequency waves of the same amplitude. However, unlike characteristic strain, the area between the source and detector curves is no longer simply related to the SNR.

\subsection{Relating the different descriptions}
The dimensionless energy density in GWs $\Omega_{\mathrm{GW}}$, spectral energy density $S_{\mathrm{E}}$, one-sided PSD, $S_{h}$, characteristic strain $h_\mathrm{c}$ and frequency-domain strain $\tilde{h}(f)$ are related via
\begin{equation}\label{eq:differentdescriptions}
H_0^2\Omega_\mathrm{GW}(f)= \frac{8 \pi G}{3 c^{2}} fS_{\mathrm{E}}(f) = \frac{2\pi^2}{3} f^3 S_h(f) = \frac{2\pi^2}{3} f^2 h_\mathrm{c}^2(f) = \frac{8\pi^2}{3} f^4 \left|\tilde{h}(f)\right|^2\; ,
\end{equation}
using (\ref{eq:strain-hc}), (\ref{eq:ShforSources}), (\ref{eq:spectralenergydensity}), (\ref{eq:omega}) and (\ref{eq:crit-density}).
Corresponding expressions for the noise are obtained by substituting $S_{n}(f)$ for $S_h(f)$, $h_{n}(f)$ for $h_\mathrm{c}(f)$ and $\tilde{n}(f)$ for $\tilde{h}(f)$. 

